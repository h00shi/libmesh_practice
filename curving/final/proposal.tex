%% -------------------------------------------------------
% Preamable ---------------------------------------------
%--------------------------------------------------------
\documentclass[letter,12pt]{article}
%-------------------------------------------------------
%\usepackage [colorlinks, linkcolor=blue,
% citecolor=magenta, urlcolor=cyan]{hyperref}   % pdf links
\usepackage {hyperref}
\usepackage{tgtermes}
\usepackage[parfill]{parskip}
\usepackage{graphicx} % inserting image
\usepackage{setspace} % double/single spacing
\usepackage[top=2cm,right=2cm,bottom=2cm,left=2cm]{geometry} % margins 
\usepackage{amsmath}  %math formulas
\usepackage{array}    %beautiful tables
\usepackage[export]{adjustbox} %aligning images
\usepackage{algorithm}
\usepackage{algorithmic}
\usepackage{color,soul} %highlight
\usepackage{subcaption}        %figures with subfigures
\usepackage{amssymb}         %fonts for N, R, ...

\graphicspath{{img/}} %where the images are

\newenvironment{tight_itemize}{
\begin{itemize}
  \setlength{\itemsep}{0pt}
  \setlength{\parskip}{0pt}
}{\end{itemize}}

\newenvironment{tight_enumerate}{
\begin{enumerate}
  \setlength{\itemsep}{0pt}
  \setlength{\parskip}{0pt}
}{\end{enumerate}}


\newcommand*\rfrac[2]{{}^{#1}\!/_{#2}}
\newcommand*{\hvec}[1]{\mathbf{#1}}
\newcommand*{\hmat}[1]{\mathbf{#1}}

% -------------------------------------------------------
% Author and title---------------------------------------
%--------------------------------------------------------
\title{Part C: Final Project Proposal \vspace{-3ex}}
\date{}
\author{}

% -------------------------------------------------------
% Begin document-----------------------------------------
%--------------------------------------------------------
\begin{document}
\maketitle
\pagenumbering{gobble}

% -------------------------------------------------------
% Intro      --------------------------------------------
%--------------------------------------------------------

High-order-accurate methods for simulation of fluid flow problems, are
among the highly pursued research frameworks in Computational Fluid
Dynamics, due to their potential to reduce the computational effort
required for a given level of solution accuracy. My
current thesis project is to generalize our in house high order
turbulent viscous flow finite volume solver\cite{jalali} to handle
three dimensional geometries.

Conventional mesh generators create linear cells near the curved
boundaries. However, high order numerical methods must account for the
curved boundary in order to maintain their order of accuracy. For an
isotropic mesh curving the faces on the boundary is sufficient. On the
other hand, for anisotropic meshes, which are common in turbulent flow
simulations, the mesh deformation should be propagated inside the
domain to prevent unacceptable self intersecting elements.

To convert a three-dimensional anisotropic linear element mesh to a
boundary conforming curved mesh, a solid mechanics analogy is
used. The initial linear mesh is modeled as an elastic solid. When the
boundary of the solid is deformed to match the curved boundary, the
internal deformation of the solid will create the desired conforming
curved mesh.

% -------------------------------------------------------
% Project      ------------------------------------------
%--------------------------------------------------------

The equations modeling the elastic solid, i.e. Navier Equations, are 
written as\cite{lai}:
%
\[
\nabla \cdot \sigma = 0
\]
%
Where $\sigma$ is the symmetric three dimensional stress tensor and
related to the displacement vector through the formulas :
%
\begin{align*}
& \sigma_{11} = (2 \mu + \lambda) u_x + \lambda v_y +  \lambda w_z \\
& \sigma_{22} = \lambda u_x + (2 \mu + \lambda) v_y +  \lambda w_z \\
& \sigma_{33} = \lambda u_x + \lambda v_y +  (2 \mu + \lambda) w_z \\
& \sigma_{12} = \mu u_y + \mu v_x \\
& \sigma_{13} = \mu u_z + \mu w_x \\
& \sigma_{23} = \mu v_z + \mu w_y
\end{align*}
%
Where $u$, $v$ and $w$ denote the components of the displacement
vector, $\mu$ and $\lambda$ are the Lame's coefficients, and the
subscripts denote partial differentiation. The Lame's coefficients are
among the properties of the solid, and will be considered constant
throughout the domain.

My proposed project is to solve this system of equations in three
dimensions using the finite element method, as a starting step for a
code that can convert linear mixed element three-dimensional meshes
into curved boundary conforming ones. Based of our current
two-dimensional mesh curving code we know that the Galerkin Method
with Cubic Lagrange Elements is a suitable approach to solve this
system of equations, and that affine mappings will work for all the
mesh elements, as the boundary of the initial domain is the actual
linear mesh.

Initially I was hoping to use libMesh\cite{kirk} for the project, but
it seems that incorporating cubic elements in libMesh is more
challenging than I thought. So I will probably code up the solver
myself in C++ and use Petsc\cite{balay} to solve the linear system of
equations resulting from the discretization.

The following stages are proposed for the project:
%
\begin{itemize}
\item[WU] For the warm up stage, I will solve a simpler scalar
  version of the problem. A Poisson problem with a manufactured
  solution in a simple geometry, e.g. box, cylinder, or sphere, would
  be a suitable candidate. It should be noted that linear mappings for
  boundary elements are enough for our purpose, and will be used for a
  cylinder or a sphere. Although the final goal is to use cubic
  elements, it is wise to implement linear and quadratic elements as
  well for verification purposes.
%
\item[A] An anisotropic mesh consisting of only tetrahera will
  be curved at this stage. As numerically describing arbitrary curved
  boundaries can become quite challenging, a simple geometry of two
  concentric spheres will be considered. Also, the method of
  manufactured solutions will be used to verify that the Navier
  Equations are being solved correctly.
%
\item[B] Two paths can to be followed here:

 In case I write the code myself I will consider generalizing the code
 to handle mixed element meshes for this part, as they are quite a
 must in solving turbulent Navier-Stokes Equations. I will add support
 for hexahedra in this part, and if things go well five faced prisms
 and pyramids will also be implemented.

 In case I can get libMesh to work and use it for this project, using
 different elements will be quite easy, thus I will curve a more
 complicated and practical geometry like a three-dimensional airfoil
 for this part of the project. Defining the three-dimensional geometry
 and the boundary conditions for the equations are the main challenges
 of this part.
%
\end{itemize}

% -------------------------------------------------------
% bibliography-------------------------------------------
%--------------------------------------------------------
\bibliography{proposal}{}
\bibliographystyle{plain}

\end{document}
